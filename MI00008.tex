%% 美赛模板:正文部分

\documentclass[12pt]{article}  % 官方要求字号不小于 12 号,此处选择 12 号字体
% 本模板不需要填写年份,以当前电脑时间自动生成
% 请在以下的方括号中填写队伍控制号
\usepackage[MI00008]{easymcm}  % 载入 EasyMCM 模板文件
\problem{B}  % 请在此处填写题号
 \usepackage{mathptmx}  % 这是 Times 字体,中规中矩 
%\usepackage{mathpazo}  % 这是 COMAP 官方杂志采用的更好看的 Palatino 字体,可替代以上的 mathptmx 宏包
\title{ Color Revolutions Are Bound To Fail}  % 标题
\usepackage{enumerate}



% 如需要修改题头(默认为 MCM/ICM),请使用以下命令(此处修改为 MCM)
%\renewcommand{\contest}{MCM}
\usepackage{bm}
\usepackage{cleveref}
\usepackage{ulem}
\usepackage{geometry}
\usepackage{framed} 
\usepackage{mathrsfs}
\usepackage{lettrine}
\usepackage{wasysym}
% 文档开始
\begin{document}

% 此处填写摘要内容
\begin{abstract}
   At the beginning of the 21st century, the wave of "color revolutions" quickly swept through the post-Soviet space and caused widespread political turmoil among Eurasian countries. Over the years, academic circles at home and abroad have conducted relatively in-depth research on the rise and spread of the "color revolution". However, looking at the research results, it seems that the relationship between color revolutions and social stability has not been explored, which will be briefly discussed in this article.

First of all, based on the research results of predecessors, supported by a large number of literature, we continue to view the grading of indicators, find 48 third-level indicators that affect social stability factors, classify them, determine their second-level indicators, and then divide them into police indicators, warning indicators and police source indicators.

Immediately afterwards, we first determined the weights of the first-level indicators and second-level indicators according to the literature analysis method, and since the sources of the third-level indicators are different, we introduce the concept of confidence, score them by the undetermined measure theory model, and then multiply the weights of the third-level indicators with their corresponding higher-level indicators to obtain the weight vectors of 48 third-level indicators. Then, in order to obtain whether there is a correlation between the three levels of indicators, we find that the DEMATEL-ISM model can handle it more conveniently, because we do not have high requirements for the confidence of the indicator to see the correlation, so the resulting Boolean matrix is enough to show the correlation between them.

The next step is to establish an early warning model for social stability. Following international practice, we have divided the warning Level into four levels: Green Level, Blue Level, Yello Level and Red Level. In this way, we can calculate the SRD Value by calculating the scores of various indicators and compare it with the control, so that we can assess the social stability level and even make an early warning.

Finally, we discussed the correlation between color revolutions and social stability, and came to a preliminary conclusion: the collapse of the social stability system is a sufficient and unnecessary condition for the success of color revolutions. We have selected two of the most representative color revolutions: Ukraine and Belarus, and believe that the fundamental reason for the failure of the Belarusian color revolution is that it did not undermine its social stability, and that the main reason for the success of the Ukrainian color revolution leading to regime change was the bloodless intervention of the US-led NATO forces in its regime.

In conclusion, we offer our views on how to prevent the color revolution in combination with the literature (see the MEMO for the details) , pointing out the strengths and weaknesses of our theoretical model and what future researchers need to continue to do.
    
    \vspace{5pt}
    \textbf{Keywords}:Unascertained Measure Model, DEMATEL-ISM Model, Early Warning Model, Color Revolution

\end{abstract}

\maketitle  % 生成 Summary Sheet
\tableofcontents  % 生成目录


% 正文开始
\section{Introduction}
\subsection{Problem Background}
China has been undergoing a period of political, economic, social, and cultural system transformation since the reform and opening up. At the same time that urbanization has accelerated and various new and old types of contradictions and conflicts have persisted, the functioning of the social system has been disrupted, which has forced society to transform in an unbalanced and uncoordinated way. The complexity and unpredictability of society have substantially expanded in the age of worldwide informationization, along with the rapid growth of the economy, and risk factors have multiplied around people's everyday lives, altering how they perceive social security and stability. 

The brutal terrorist attack in Kunming, the explosion accident in Tianjin Binhai New Area, the Changchun Changsheng vaccine incident, and the repeated outlawing of environmental pollution and economic aid crimes all strongly suggest the emergence of a risk society. As a result, China must now grapple with the issue of how to preserve social harmony and stability while the country is developing economically and socially and reduce the negative effects of multiple risk events.

Since the beginning of the new century, the international and regional situation has been turbulent, and the ``color revolution'' is like a sword of Damocles hanging above the Eurasian continent, especially the post-Soviet space, and the Western forces led by the United States have never given up their attempt to carry out ``democratic transformation'' of the political system of the former Soviet republics. This poses a huge threat to Eurasian countries, including China and Russia.

From 2003 to 2005, ``Color revolutions'' occurred in Georgia, Ukraine, and Kyrgyzstan, three neighboring countries of Russia, forcing regime change in a relatively non-violent manner, his original Union Republic of the Soviet Union caused varying degrees of shock. This new type of political movement is obviously different from the traditional forms of war, and has become one of the important means by which the western forces, led by the United States, forcibly intervene in the internal affairs of other countries and try to overthrow the regime, it poses a serious challenge to the security situation in the region.

In the past two decades, the ``Color revolution'' has not been abated. Instead, it has been constantly upgrading technological means and tools, with a lasting impact on the security of the Eurasian region. Objective results show that: first, the threat of ``Color Revolution'' can not be underestimated, and may even be repeated in many parts of the world, is a long-term follow-up attention, and second, the ``Color Revolution'' is full of hypocrisy and lies. Many countries that have experienced the ``Color revolution'' have not been able to achieve the desired ``Democratic transformation'' to bring about fruitful results, it is mired in economic ruin and political chaos. The so-called western-style ``Democracy'' is only to foster pro-western political forces to come to power, thus providing more economic and political benefits for the west. It is not to really solve local social and economic problems, the ultimate victim is the local people.
\subsection{Problem Restatement}
\begin{itemize}
\item \textbf{For problem 1:} We are required to select representative indicators in various aspects to reflect various aspects of social stability. And from a qualitative and quantitative point of view, establish a system of indicators of social stability, and discuss their interrelationship.
\item \textbf{For problem 2:} We need to establish an early warning model for social stability based on the indicator system that affects social stability established in the first question, and discuss it.
\item \textbf{For problem 3:} We need to select a country or region that has had a color revolution and use the indicator system and early warning model established by our first and second questions to assess its social stability. And find out the main reasons or factors for the failure of this color revolution, judge the trend of future social stability, and put forward our suggestions.
\item \textbf{For problem 4:} Find a country where color revolutions have led to regime change, and use the model we have built to find the main causes of regime change.
\item \textbf{For problem 5:} Put forward our team on the prevention of color revolution, maintain social stability recommendations.
\end{itemize}
\subsection{Our Work}
Previous research in this area has been very in-depth, and domestic research on the social stability index system is endless. Only then can we use its research results to solve the problems we encounter. Here's our work:

{\LARGE\CheckedBox} Based on the theory of risk society and social governance, this paper summarizes the relevant literature of social stability and its measurement, and discusses the construction of social stability index and index system. 

{\LARGE\CheckedBox} Based on the research and analysis of social stability risk sources, a dimensional model of social stability index was constructed. Then, according to the principles of data availability, scientificity, and operability, specific indicators are set and selected, and a complete index system framework is gradually built to analyze the actual operation and results of the social stability index.

{\LARGE\CheckedBox} Based on the determination of various indicators and weight distribution models of the social stability index, the research objects that have occurred in the color revolution were selected to evaluate their social stability, and the main risk factors affecting social stability in the color revolution were identified.
\begin{figure}[htbp]
\centering
\includegraphics[width=.4\textwidth]{img/our work.pdf}
\caption{Our Work}
\end{figure}
\section{Notations}
The primary notations used in this paper are listed in Table \ref{tb:notation}.

% 三线表示例
\begin{table}[!htbp]
\begin{center}
\caption{Notations}
\begin{tabular}{cc}
	\toprule
	\multicolumn{1}{m{3cm}}{\centering Symbol}
	&\multicolumn{1}{m{8cm}}{\centering Definition}\\
	\midrule
	$A_i$&Level 1 indicators\\
	$A_{ij}$&Level 2 indicators\\
	$A_{ijk}$ &Level 3 indicators\\
	$SRD$&Social Risk Degree\\
	\bottomrule
\end{tabular}\label{tb:notation}
\end{center}
\end{table}













\section{Establishment of Our Model}
\subsection{Unascertained Measure Model}
Let $r_1,r_2,\cdots ,r_m$ be $m$ objects to be optimized, then $\bm{R}=\{r_1,r_2,\cdots ,r_m\}$ can represent the space to which the object to be optimized belongs.
Each $r_i(i = 1,2,\cdots ,m)$ consists of $n$ evaluation indexes. Recorded as $t_1, t_2,\cdots ,t_n$. 
Use $\bm{T}=t_1, t_2,\cdots ,t_n$ to represent the evaluation index space of $r_i$ ,then $r_i$ can be expressed as an n-dimensional vector $\bm{r_i}=\{r_{i1},r_{i2},\cdots ,r_{in}\}$. 
The observed value of the evaluation index is expressed by $r_{ij}(i=1,2,\cdots , m; j=1,2,\cdots ,n)$. 
Suppose that each $R_{ij}$ has $p$ evaluation levels, denoted $C_1,C_2,\cdots ,C_p$. 
Then the overall evaluation space can be denoted as $\bm{C}={c_1,c_2,\cdots ,c_p}$. 
Among them, the $k$ evaluation grade can be expressed by $c_k(k=1,2,\cdots ,p)$, and if the $k$ level is greater than the $k+1$ level, it is written as $c_k>c_{k+1}$.
If there is $c_1>c_2>\cdots >c_p$ or $c_1<c_2<\cdots <c_P$, then $c_1, c_2,\cdots ,c_p$ is an ordered division class.
\begin{enumerate}
    \renewcommand{\labelenumi}{\textbf{Step \theenumi}}
    \item Measurements $r_{ij}$ of the $k$ rating level $c_k$. The degree is expressed asin the $u_{ijk}=u(r_{ij}\in c_i)$, requirement $u$ to meet:
\begin{eqnarray}
    &0\le u(r_{ij}\in c_k)\le 1
    \label{e1}\\
    &u(r_{ij}\in C)=1
    \label{e2}\\
    &u(r_{ij}\in U_{i=1}^kc_i)=\sum_{i=1}^{k}u(r_{ij}\in c_i)\label{e3}
\end{eqnarray}
Thereinto, $i=1,2,\cdots ,m$

\hspace{4.25em}$j=1,2,\cdots ,n$

\hspace{4.25em}$k=1,2,\cdots ,p$.\\
Thereinto, Equation \ref{e1} represents ``non-negative boundedness'', Equation \ref{e2} represents ``normalization'', Equation \ref{e3} represents ``additivity''. The $u$ that simultaneously satisfies the Equation \ref{e1} to Equation \ref{e3} is called an unascertained measure, and has a single index measure matrix $(u_{ijk})_{n\times p}$:
\begin{equation}
    (u_{ijk})_{n\times p}=\begin{bmatrix}
  u_{i11}&u_{i12}  &\cdots  &u_{i1p} \\
 u_{i21} &u_{122}  &\cdots  &u_{i2p} \\
\vdots  & \vdots   & \ddots &\vdots  \\
  u_{in1}&u_{in2}  &\cdots  &u_{inp}
\end{bmatrix}
\end{equation}
\item According to the grading standard of the evaluation index and the measured value of each index, the comprehensive measurement evaluation matrix of the index is determined, and the comprehensive measure of multiple indicators is calculated in combination with the comprehensive index weight determined by the improved entropy weight method, which is as follows (Wang Xinmin et al., 2012):
\begin{equation}
    u_{ik}=\sum_{j=1}^{n}w_ju_{ijk}
\end{equation}
Thereinto, $i=1,2,\cdots ,m$

\hspace{4.25em}$j=1,2,\cdots ,n$

\hspace{4.25em}$k=1,2,\cdots ,p$.
\item Calculate $u$ as:
\begin{eqnarray}
    &0\le u_{ik}\le 1\\
    &u(r_i\in C)=\sum_{k=1}^{p}u_{ik}=1\\
    &u(r_i\in \bigcup_{l=1}^{k}c_l)=\sum_{l=1}^{k}u(r_i\in c_l)
\end{eqnarray}
Therefore, the object of evaluation is obtainedriThe p-dimensional vector of the multi-index comprehensive measure of , which can be expressed as $\bm{U}=\{u_{i1},u_{i2},\cdots ,u_{ip}\}$. 
\item Multi-indicator evaluation matrix $(u_{ik})_{m\times p}$ as follows:
\begin{equation}
    (u_{ik})_{m\times p}=\begin{bmatrix}
 u_{11} &u_{12}  &\cdots   &u_{1p} \\
  u_{21}&u_{22}  & \cdots  &u_{2p} \\
  \vdots & \vdots &  \ddots & \vdots\\
 u_{m1} &u_{m2}  &\cdots   &u_{mp}
\end{bmatrix} 
\end{equation}
    \item In order to determine the weight of each indicator, the evaluation level of the gold cave tailings pond was calculated by using the confidence identification criterion. Set the reliability to $\lambda$ and $\lambda \ge 0.5$. If $c_1>c_2>\cdots >c_p$, its recognition model is:
    \begin{equation}
        \min \{k:\sum_{l=1}^{k}u_{il}\ge \lambda,1\le k\le p\}
    \end{equation}
Thereinto,$k=1,2,\cdots ,p$,$s$ is the degree of affiliation. When the value of $k$ satisfies the recognition model, the membership degree $s$ is calculated to obtain the evaluation objectriBelongs to the $s$ rating and is credited as $c_s$.
    \item After deriving the security level of the evaluated object according to the confidence identification criterion, it is also necessary to rank the degree of influence of the influencing factors. If the ordered evaluation space is $\{c_i\}$, then $c_k$ of the value equals to $e_k$, and $e_k>e_{k+1}$. Then we have:
    \begin{equation}
        q_{Bi}=\sum_{k=1}^{p}e_ku_{ik}
    \end{equation}
Thereinto,$q_{Bi}$ is the importance of the unascertained measure, and the importance vector of the unascertained measure $q=\{q_{B1}, q_{B2},\cdots , q_{Bn}\}$. The influence degree of the influencing factors is ranked by comparing the size of \rm{$q$}.
\end{enumerate}
\subsection{ISM Model}
ISM (Interpretive Structural Modeling) is a model that is developed to study complex systems. 
Based on tools such as directed graph, matrix and computer technology, a multi-level 
hierarchical structure model is constructed (POLAT \& RMAC, 2011, p. 169-174). DEMATEL 
(Decision Making Trial and Evaluation Lab), which is a scientific method based on graph theory 
and matrix to simplify the complex system structure (Gu Xuesong \& Chi Guotai, 2010, p. 508-
514). The combined model in this paper integrates the centrality and causation of DEMATEL 
into the multi-level hierarchical structure of ISM, which can not only clarify the hierarchical 
relationship of various influencing factors but also study the relative importance of constraints, 
so as to make the analysis result more objective and reasonable.
The steps to build the composite model are as follows:
\begin{enumerate}
    \renewcommand{\labelenumi}{\textbf{Step \theenumi}}
    \item Determine the set of influencing factors :
\begin{equation}
A=\{a_i|i=1,2,\cdots ,n\}
\end{equation}
    \item Determine the factor influence scale, and determine the mutual influence relationship between 
the factors through expert knowledge and experience, and get the direct influence matrix $V$.
\begin{equation}
    V=[v_{ij}]_{n\times n}
\end{equation}
Thereinto, $v_{ij}$ represents the influence degree of factor $a_i$ on factor $a_j$. When $i=j$ , $v_{ij}=0$.
    \item  Calculate the direct impact matrix $V$ to obtain the normalized direct impact matrix $X$ :
    \begin{equation}
X=[X_{ij}]_{n\times n}=\frac{V}{\max \sum_{j=1}^{n}V_{ij}}
\end{equation}
    \item Calculate the comprehensive impact matrix $T$ :
\begin{equation}
T=[T_{ij}]_{n\times n}=X(I-X)^{-1}
\end{equation}
Thereinto, $I$ is identity matrix.
    \item The influence degree $f_i$ , the influence degree $e_i$ , the center degree $z_i$ and the reason degree $y_i$ of 
the constraint factors were calculated. The calculation formula is as follows:
\begin{eqnarray}
    &f_i=\sum_{j=1}^{n}T_{ij},1\le i\le n \\
    &e_i=\sum_{j=1}^{n}T_{ij},1\le i\le n\\ 
    &z_i=f_i+e_i\\
    &y_i=f_i-e_i
\end{eqnarray}
    \item Draw the cause and result diagram:

Cartesian coordinate system is drawn with the degree of center as the abscissa and the degree of 
cause as the ordinate.
    \item  Calculate the overall impact matrix $H$ :
    \begin{equation}
        H=[H_{ij}]_{n\times n}=T+I
    \end{equation}
    \item Determine the threshold value $\lambda$ (Xue Wei1 \& Geng Zhiwei, et al. 2019, p. 99-104.):
\begin{equation}
    \lambda =\alpha +\beta
\end{equation}
Where, and respectively refer to the mean value and standard deviation of the comprehensive 
influence matrix $T$ .Different $\lambda$ values have different logical relationships with the 
influencing factors (Sun Jing, 2018). The choice of $\lambda$ is more subjective based on expert 
experience, while replacing it with the sum of the mean and standard deviation based on the 
statistical distribution is more objective, which can reduce the influence of subjectivity.
    \item Calculate the standardized reachable matrix $K$ :
\begin{equation}
    K=[K_{ij}]_{n\times n}
\end{equation}
Thereinto, if $H_{ij}>\lambda$ ,then $H_{ij}=1$ 

\hspace{4.35em}if $H_{ij}\le \lambda$ ,then $H_{ij}=0$ .
\item  According to the reachability matrix, the reachability set $R_i$ and antecedent set $S_i$ of each 
influencing factor are determined.\\ 
Thereinto, $R_i$ is composed of the index set corresponding to all the 
columns with index 1 in the ith row of the reachable matrix

\hspace{4.25em} $S_i$ consists of the set of indices 
corresponding to all rows with index 1 in the ith column of the reachable matrix.
\item Verify:
\begin{equation}
    R_i=R_i\cap S_i,(i=1,2,\cdots ,n)
\end{equation}
If it is true, then $a_i$ is the highest level factor. At this time, row $i$ and column $i$ are deleted in $K$ , 
and the calculation is repeated until all factors are deleted.
\item Draw the hierarchical structure diagram of factors according to the order of factors to be 
deleted, and establish the structural model.
\end{enumerate}
\subsection{DEMATEL Model}
The decision experiment and evaluation laboratory method, or DEMATEL, is a mathematical language for quantifying complex system problems by using graph theory and matrix tools. DEMATEL obtains the degree of centrality and the degree of cause by calculating the degree of influence and the degree of being affected, and then analyzes the dependence among the factors. The steps for the DEMATEL method are as follows.
\begin{enumerate}
    \renewcommand{\labelenumi}{\textbf{Step \theenumi}}
\item The object factors are determined, and the direct influence matrix $X$ is established according to the logical relations among the factors.

\item Matrix normalization process, sum the rows of matrix $X$, set its value to $Sum_i (i=1,2,\cdots ,n)$, find the maximum value $Sum_{max}$, let $X'=X/Sum_{max}$, get the normalized matrix $X'$.
\item Calculating the comprehensive influence matrix $T$, calculating the matrix $T$ according to the formula $T = X'(I-X')^{-1}$, where $I$ is the unit matrix.
\item The influence degree ($T$), the affected degree ($R$), the center degree ($P$) and the reason degree ($E$) were calculated according to the comprehensive influence matrix $T$.
\item Dawing the distribution map of the influencing factors according to the centrality and the degree of cause.
\item The causal factor group and the result factor group were analyzed iteratively, and the causal hierarchy diagram was drawn.
\end{enumerate}
\subsection{Establishment of DEMATEL-ISM Model}
DEMATEL-ISM was proposed by American scholars. By combining adjacency matrix and direct influence matrix, this method decomposes the complex system into multi-level hierarchical form with clear hierarchy, quantifies the risk factors, studies the influence degree and affected degree of risk factors, and obtains the hierarchical structure relationship of risk factors.

DEMATEL-ISM combines DEMATEL and ISM theories, which can effectively determine the causal relationship between factors, obtain the hierarchical structure of influencing factors, excavate the deep-seated factors leading to accidents, and thus provide a theoretical basis for the proposal of accident prevention measures.

In order to fully analyze the influencing factors of social stability, the DEMATEL-ISM method is specially used to analyze the key factors and core factors that cause accidents, providing theoretical support for preventing accidents. The specific process is as follows: build the impact matrix, determine the impact strength between the factors affecting the fire and explosion accidents in the laboratory, determine the direct impact matrix and normalize it. 
\begin{enumerate}
    \renewcommand{\labelenumi}{\textbf{Step \theenumi}}
\item According to Unascertained Measure Model and ISM Model, the intensity of action between the influencing factors was analyzed, and the values were assigned according to 5 levels, including no influence 0, small impact 1, average impact 2, large impact 3 and severe impact 4, and two initial direct impact matrices were obtained. To eliminate the fluctuation between fractions, the 2 direct impact matrices are averaged to obtain the direct impact matrix $W$.
\item The row value maximum method is used to process the direct impact matrix to obtain the normalized matrix $N$:
\begin{eqnarray}
	&Maxvar=\max (\sum_{j=1}^{n}a_{ij})\\
	&N=(\frac{a_{ij}}{Maxvar})_{n\times n}
\end{eqnarray}
\item Calculated the comprehensive impact matrix $T$ according to the Equation \ref{e4}:
\begin{equation}
	T=(N+N^2+N^3+\cdots +N^k)\label{e4}
\end{equation}
Thereinto, $N\times N$ means the indirect relationship of increase, which includes the increase between the values that are not 0 in the direct impact matrix, and the transfer of the influence between the elements causes the value of 0 to become a non-zero value.
\item According to the data distribution of the overall influence matrix $E$, the threshold $\lambda$ is determined and the factors with less influence are screened out, so as to construct the up matrix $M$. According to the reachability matrix $M$, solve for the reachability set $P_{(S_i)}$, the antecedent set $Q_{(S_i)}$ and the common set $C_{(S_i)}=P_{(S_i)}\cap Q_{(S_i)}$ of each factor. According to the principle of hierarchical processing, when $L_1=C_{(S_i)}=P_{(S_i)}$, $S_i$ is the first layer element, and then the rows and columns corresponding to the first layer factor are crossed out.
\item Repeat the operation until all the elements are divided, thus obtaining a multi-level directed topological graph between the factors:
\begin{eqnarray}
	&E=T+I\\
	&M=(m_{ij})_{n\times n},m_{ij}=\left\{\begin{matrix}
 1 &m_{ij}\ge \lambda \\
 0 &m_{ij}<\lambda
\end{matrix}\right.
\end{eqnarray}
Thereinto, $M_{ij}$ corresponding value in reachable matrix $M$. Here, we take $\lambda=0.15$ to judge.
\end{enumerate}
\subsection{Establishment of Early Warning Model}
\label{ewm}
\begin{enumerate}
	\renewcommand{\labelenumi}{\textbf{Step \theenumi}}
\item We have established an indicator system that affects social stability and determined the weights between each indicator, so that we can obtain the following formula for calculating the degree of social risk:
\begin{equation}
	SRD=\sum I_nW_n
	\label{eq}
\end{equation}
Thereinto, $SRD$ represents the degree of social risk

\hspace{4.25em}$n$ is the serial number of the indicator and its weight

\hspace{4.25em}$I$ represents the indicator

\hspace{4.25em}$W$ represents the weight of the indicator in the entire social risk early warning indicator system.
\item Each indicator in the indicator system uses a five-level scoring method, that is, five values are set according to the size of the indicator value: 10, 20, 30, 40 and 50. The size of the indicator value is directly proportional to the degree of social risk. In this way, we can measure the degree of social risk through the above formula for calculating the degree of social risk and identify it with corresponding early warning signals.

We scored the indicators based on the data we got and calculated the Social Risk Degree(SRD).By consulting the data, we divide the warning level,there are shown in Table \ref{srd}:
\begin{table}[!ht]  
    \centering
    \caption{Weighted Comprehensive Assessment of Social Risk Police Ranks}
    \label{srd}
    \begin{tabular}{ccccc}
    \hline
        SRD Value & [10,20) & [20,30) & [30,40) & [40,50] \\\hline 
        Alarm Level & No Alarm & Light Alarm & Medium Alarm & Heavy Alarm \\ 
        Signal Level& Green Level& Blue Level& Yello Level& Red Level\\ \hline
    \end{tabular} 
\end{table}
\end{enumerate}




































\section{Solution For Problem One \& Two}
\subsection{Selection of Indicators}
The early-warning mechanism of social security and stability refers to the critical state that signals the operation of society and shows that disorderly phenomena have taken place or are about to take place, a set of systems and methods aimed at attracting the attention of policy makers, managers and the public, analyzing the causes in a timely manner, and implementing effective measures so as to prevent the undesirable phenomena of social operation from further worsening. In the face of the rapid development of modern society, it will be helpful for government decision-making departments and public security organs to establish a complete and effective early-warning mechanism for social security and stability, guan took timely and effective preventive measures against risks in social development in order to maintain and promote social harmony and stability.

Social security is a more complex concept, and due to the broad nature of its content, there is a distinction between broad and narrow senses. Social security in the broadest sense refers to the state of social operation in which the entire social system can maintain benign operation and coordinated development, and the insecurity factors and influence are minimized. Obviously, social security includes economic security, political security, social life security, ideological and cultural security, and many other aspects. Social security in the narrow sense refers to security in areas other than economic and political systems. Based on the analysis of the above two aspects, we believe that the connotation of social security can be interpreted as: Social security refers to the security of the public living space of the population, which includes the security of citizens' lives and property, the order of social life, and the ecological environment, and it directly reflects the needs of public security interests closely related to citizens.

We believe that there is no absolute objectivity and reasonableness in the selection of indicators, of course, we cannot guarantee that the indicators we choose will be reasonable, but we read a lot of literature to ensure that our indicators are as objective and reasonable as possible within our ability.

Social stability includes political stability, economic stability, normal social order and people's peace of mind. These aspects are interrelated, mutually influencing, interacting and inseparable. Political stability is the core of social stability as a whole, economic stability is the foundation of social stability as a whole, normal social order is a necessary condition for political stability and economic stability, and people's peace of mind is a comprehensive reflection of social stability. We divide the indicators into three categories, police source indicators, warning indicators and alarm indicators. According to the analytic hierarchy method, the indicator system is divided into target layer($A_i$), criterion layer($A_{ij}$) and indicator layer($A_{ijk}$). The specific meaning of each of these indicators is listed in Figure \ref{zhibiao}.

Since the standard layer is all consistent, there is a certain correlation between risk factors, so it is necessary to analyze the correlation between social stability risk factors.

We have included our indicators in the Figure \ref{soi}:
\begin{figure}[htbp]
\centering
\begin{subfigure}[b]{.32\textwidth}
\includegraphics[width=\textwidth]{img/1.pdf}
\end{subfigure}
\begin{subfigure}[b]{.32\textwidth}
\includegraphics[width=\textwidth]{img/2.pdf}
\end{subfigure}
\begin{subfigure}[b]{.32\textwidth}
\includegraphics[width=\textwidth]{img/3.pdf}
\end{subfigure}
\caption{Selection of Indicators}\label{soi}
\end{figure}
\subsection{Solution For Problem Two \& Establishment of Indicator System}
Here in order to more intuitively show the two matrices we have obtained, we will show the \textbf{Direct Influence Matrix} \bm{$W$}, \textbf{Comprehensive Influence Matrix} \bm{$T$} and the \textbf{The Reachability Matrix} \bm{$M$} in the form of Heatmap in the Figure \ref{heatmap}:
\begin{figure}[htbp]
\centering
\begin{subfigure}[b]{.32\textwidth}
\includegraphics[width=\textwidth]{img/h2.eps}
\caption{Matrix $W$}
\end{subfigure}
\begin{subfigure}[b]{.32\textwidth}
\includegraphics[width=\textwidth]{4}
\caption{Matrix $T$}
\end{subfigure}
\begin{subfigure}[b]{.32\textwidth}
\includegraphics[width=\textwidth]{kdjz}
\caption{Matrix $M$}
\end{subfigure}
\caption{Heatmaps of Each Matrixes}
\label{heatmap}
\end{figure}

From the Heatmap above, we can easily see the relationship between the three indicators (1 in the matrix $M$ indicates that there is a relationship, 0 indicates that there is no relationship). According to our model, there is a positive correlation between the three indicators and the two indicators (we have carried out a positive treatment on the indicators).

At this time, the \textbf{Index System} we get is: with the \textbf{Alert Indicator($A_1$, Weight 0.390)}, \textbf{Warning Indicator($A_2$, Weight 0.319)}, and \textbf{Alarm Indicators($A_3$, Weight 0.291)} as the Level 1 indicators. Their secondary indicators are \textbf{Accident Disaster($A_{i1}$, Weight 0.123), Social Crisis($A_{i2}$, Weight 0.245), Public Health Events($A_{i3}$, Weight 0.491) }and \textbf{Natural Disaster($A_{i4}$, Weight 0.142)} respectively. Level 3 indicators and their weights are listed in the Table \ref{zhibiao}:
\newpage 
\begin{table}[ht]
%\small
    \centering
    \caption{The Meaning and Weight of Each Indicator}
    \label{zhibiao}
    \resizebox{\textwidth}{!}{
    \begin{tabular}{ccc}
        \hline
        Indicator & Indicator Layer &Weight\\ \hline
        A111 & Enterprise Loss Degree &0.014 \\ 
        A112 & Lack Of Investment In Urban Infrastructure &0.035 \\ 
        A113 & The Degree Of Consistency Between The Urban Environment And Production &0.056 \\ 
        A114 & The Degree To Which Urban Policies Pay Attention To Production Safety &0.018 \\ 
        A121 & Crime Rate &0.030 \\ 
        A122 & Divorce Rate &0.099 \\ 
        A123 & Income Difference Degree Of Urban Residents &0.093 \\ 
        A124 & Urban Real Unemployment Rate &0.023 \\ 
        A131 & Development Degree Of Urban Health Sector&0.093  \\ 
        A132 & Concern About Urban Environmental Sanitation  &0.132\\ 
        A133 & Defense Against External Public Health Events &0.206 \\ 
        A134 & The Degree Of Attention Paid By Urban Policies To Health Care &0.059 \\ 
        A141 & Extent Of Urban External Environment Damage&0.066  \\ 
        A142 & Climate Variability &0.022 \\ 
        A143 & Potential Threat To Urban Geological Structure&0.011  \\ 
        A144 &  Public Health&0.044 \\ 
        A211 & Frequency Of Production Accidents&0.048  \\ 
        A212 & Injury And Death Rate In Production Accidents&0.031  \\ 
        A213 & Damage Of Urban Infrastructure&0.010 \\ 
        A214 & Damage Degree Of Urban Infrastructure &0.034 \\ 
        A221 & Dissatisfaction With Social Order &0.049 \\ 
        A222 & Frequency Of Labor Disputes &0.057 \\ 
        A223 & Degree Of Pollution And Damage Accidents &0.080 \\ 
        A224 & Non-Institutionalized Group Development &0.059 \\ 
        A231 & Potential Occurrence And Activity Of Public Health Events &0.250 \\ 
        A232 & Active Degree Of Inducements For Public Health Events &0.184 \\ 
        A233 & Main Factors Of Natural Disasters &0.034\\ 
        A234 & Natural Disasters &0.023\\ 
        A241 & Instability Of Natural Disasters &0.055 \\ 
        A242 & Active Degree Of Main Factors Of Natural Disasters  &0.038\\ 
        A243 & Active Degree Of Natural Disaster Inducing Factors  &0.043\\ 
        A244 & Warfare Caused By Accidents And Disasters &0.007 \\ 
        A311 & Life Loss Caused By Accidents And Disasters &0.068 \\ 
        A312 & Economic Losses Caused By Accidents And Disasters  &0.055\\ 
        A313 & Group Crime &0.016 \\ 
        A314 & Group Fighting  &0.004\\ 
        A321 & Frequency And Scale Of Group Crime And Fighting&0.115  \\ 
        A322 & The Frequency And Scale Of Religious And Ethnic Conflicts &0.090 \\ 
        A323 & Active Degree Of Natural Disaster Inducing Factors  &0.008\\ 
        A324 & Warfare Caused By Public Health Events &0.009 \\ 
        A331 & Life Loss Caused By Public Health Events &0.217\\ 
        A332 & Economic Losses Caused By Public Health Events &0.159 \\ 
        A333 & Psychological Problems Caused By Public Health Events &0.115 \\ 
        A334 & Life Loss &0.003 \\ 
        A341 & Degree Of Life Loss &0.046\\ 
        A342 & Property Damage Degree&0.029  \\ 
        A343 & Degree Of Direct Production Loss&0.051  \\ 
        A344 & Indirect Loss Degree Of Natural Disasters&0.016  \\ \hline
    \end{tabular}}
\end{table}
\restoregeometry
\subsection{Solution For Problem Two}
We have established a social stability early warning model, which only needs to be scored according to the actual situation of each indicator, and the stability level can be obtained according to the final situation. But how to determine the score of each indicator we still need to discuss as follows:

Four aspects should be considered to determine the early warning level:
\subsubsection{Characteristic}
According to the predictability of its occurrence, it can be divided into sudden and recurrent. Due to their unpredictability, emergencies often have a strong impact on the people after they occur, thereby endangering social security and stability. Because regular events occur more frequently, the people have a certain ability to bear them, so when they occur, the intensity of the impact is less than that of sudden events. From the perspective of foreseeability, the warning level of sudden events should be higher than that of recurring events.

According to the relationship between its harm and residents, it can be divided into direct harm and indirect harm. As the direct hazard type is directly aimed at the personal safety of residents, it will cause great panic to residents once it occurs. The indirect hazard type will not directly threaten the personal safety of residents, so it is not easy to cause huge panic. As for the relationship between hazards and residents, the warning level of direct hazards should be higher than that of indirect hazards.

Considering the two kinds of nature of early warning events. From high to low, the early warning levels are \textbf{Sudden Direct Harm Type, Regular Direct Harm Type, Sudden Indirect Harm Type} and \textbf{Daily Indirect Harm Type}.
\subsubsection{Severity}
We have determined the severity of the incident in three respects: the level of threat to \textbf{The Lives of The Population, The Magnitude of The Economic Damage} and \textbf{The Potential of Further Harm}.
\subsubsection{Scope of Influence}
We believe that the influence scope can be measured from three aspects: \textbf{The Number of People Affected, The Impact of The Spatial Scope} and \textbf{The Impact of Psychological Degree}. The higher the number of residents involved in the incident, the higher the level of early warning; the greater the spatial scope of the incident, the higher the level of early warning; the stronger the psychological impact of the incident on the residents, the higher the level of early warning.
\subsubsection{Controllability}
We believe that people feel safe about situations they can control and fear situations they can't control, so event controllability affects the level of warning. The higher the degree of uncontrollability, the higher the warning level, and if it is completely out of control, it is extremely dangerous. Event controllability is measured from two aspects, namely, \textbf{The Understanding and Mastery of Relevant Factors} and \text{The Timely and Effective Degree of Measures}.

In the actual early warning, an event will have the above multiple attributes, so it is necessary to comprehensively analyze according to the specific situation and determine the early warning level of each indicator in the urban early warning indicator system.

When we determine the score of the indicator, we only need to bring in Equation \ref{eq} to get the warning level.
\section{Analysis of Problem Three \& Four}
In previous studies, the understanding of the color revolution has been divided to a certain extent. This article selects and lists one example of Chinese and foreign scholars:






\begin{center}
\noindent\fbox{
    \parbox{.9\linewidth}{\uline{\textbf{Zhu Xiaomin Summarized Five Characteristics of Color Revolutions:} \hfill }
\begin{enumerate}
\item It was instigated by the western forces led by the United States
\item It mainly occurs in the CIS countries where the ruling leaders have a pro-Russian tendency
\item It must be aimed at establishing a pro-American and pro-Western regime
\item They are carried out under the banner of ``democracy''
\item The ruling party and the opposition have no fundamental differences and differences between socialism and capitalism.
\end{enumerate}}
}
\end{center}
\begin{center}
\noindent\fbox{
    \parbox{.9\linewidth}{\uline{\textbf{Mark R. Beissinger Summarized Six Characteristics of Color Revolutions:} \hfill }
\begin{enumerate}
\item Using unfair elections as an opportunity for mass mobilization against pseudo-democratic regimes
\item The support of external forces for the development of the democratic movement in the country 
\item Using unconventional protest tactics before elections to organize radical youth movements in order to undermine the prestige of the regime and prepare for the final decisive battle 
\item The united opposition established through external forces 
\item external diplomatic pressure and unusually large-scale electronic regulation
\item Mass mobilization after the announcement of false election results, using Gene Sharp's tactics of nonviolent resistance
\end{enumerate}}
}
\end{center}




Before we study the relationship between social stability and color revolutions, we must first clarify what color revolutions are, and we have found such a sentence:``The essence of color revolutions is that Western forces led by the United States illegally interfere in the internal affairs of other countries through non-governmental organizations and other forms''. Here we can analyze that the color revolution itself is not a natural occurrence of social instability or social crisis as our model says. But here we analyze and find that if the society is very stable, it will never allow the United States and other countries to take advantage of it. Therefore, in order to prevent the occurrence of color revolutions, it is also necessary to analyze social stability. In the face of the infiltration of foreign forces, we must fully enhance our country's ``immunity'' against color revolutions internally.

As mentioned above, in order to study the relationship between regime change and social stability caused by color revolutions, we choose the two most representative color revolutions to analyze the specific social stability. The analysis of the color revolution in history by predecessors is quite perfect. We will not dwell on it here, but simply introduce the two revolutions as follows:

\begin{itemize}
\item \textbf{Ukraine's Orange Revolution} 
\end{itemize}
\begin{center}
\noindent\fbox{
    \parbox{.9\linewidth}{~~~~On November 21, 2004, then-Prime Minister Vichny Yanukovych won the election by a majority of votes in the second round. The opposition, also led by Yushchenko, rejected the election results as fraudulent, and a massive revolt broke out, with Yushchenko and his colleagues calling on the crowd to surround government buildings. Under the dual pressure of strong condemnation by the United States-led Western powers and domestic demonstrations, Yanukovych had to accept a second election, which Yushchenko won on December 26, 2004. The city of Kiev is decorated with orange chestnut flowers, and Yushchenko's campaign and supporters use orange as the logo color, so the "color revolution" in Ukraine is called the "orange revolution" or "chestnut flower revolution".}
}
\end{center}

\begin{itemize}
	\item \textbf{The First Belarusian Color Revolution}
\end{itemize}
\begin{center}
\noindent\fbox{
    \parbox{.9\linewidth}{~~~~In March 2006, the Belarusian presidential election, Alexander Lukashenko was overwhelmingly elected President of Belarus. Russian President Putin immediately congratulated Lukashenko and said the President of Belarus election process was legal, fair and transparent. Opposition demonstrations broke out in Minsk after the Belarusian opposition claimed Alexander Lukashenko had “Cheated” in the election, with the US and Europe rejecting the results, but the Belarusian ``Color revolution'' that the US and Europe had been waiting for was ultimately defeated.}
}
\end{center}

We will use the Social Early Warning Model we have established to score the three-level indicators in our index system according to the historical data and historical evaluation of the two color revolutions. In order to determine the trend of their SRD Value and Alarm Level. This is our discussion on internal causes, and behind it, Western forces led by the United States are playing a driving role. We will explore their impact on regime change through our social stability and how to promote Western forces. Based on this, combined with the findings of previous studies, this paper discusses the causes of failure and the main factors affecting the success of regime change caused by their color revolution.
\newpage 
\section{Solution For Problem Three \& Four}
Here we score according to the performance of each indicator during the two color revolutions, and then calculate the SRD Value according to the social stability early warning model we established, in order to visually show the changes of the two color revolution SRDs, we show it in the Figure \ref{data}:
\begin{figure}[htbp]
\centering
\includegraphics[width=\textwidth]{img/data.eps}
\caption{Changes Of Social Instability Index}
\label{data}
\end{figure}

What we do know is that the color revolution in Belarus did not lead to regime change, and that this was reflected in social stability: \textbf{During this period, the social stability of Belarus did not change too dramatically, all within the Light Alarm range, showing periodic changes around the mean}.

However, in Ukraine, the performance of regime change in the social stability index is: \textbf{The change over time showed an upward trend, the Alarm Level moved from Medium Alarm to Heavy Alarm, and the social warning model issued a red light alarm}.

Here we can initially draw an inaccurate conclusion:

\begin{center}
\noindent\fbox{
    \parbox{.9\linewidth}{\LARGE{\centering{Even a colour revolution is unlikely to lead to regime change if it does not destabilise society first.}}}
}
\end{center}

Here we think of the earth's ecosystem, under the disturbance of external factors, the stability of the ecosystem is reduced, but we all know that the ecosystem has a certain ability to regulate, so here the social stability index changes periodically to be explained: the social stability system has a certain self-regulation ability, and the color revolution can reduce social stability, but does not exceed the regulatory limit, so it changes periodically around the mean.

Analysing the social stability in Ukraine and Belarus separately, we conclude that the decisive factor was the success of the color revolution in Ukraine and the failure of the color revolution in Russia: whether the maximum tolerance limit of the social stability system was exceeded, that is, whether the current level of early warning was allowed to move the current level of warning across borders to the next warning level.

But we can't say so arbitrarily, \textbf{social stability should be a necessary condition for the success of color revolutions, not a sufficient condition}. Let's go back to the note about color revolutions: ``Driven by the West, led by the United States''. Therefore, it can be seen that the external pressure on the stable system is also one of the influencing factors: this is the main guiding position in previous studies. Our relations with Western countries such as the United States today are self-explanatory.

At this point, we have learned that the main reason for the failure of the Belarusian color revolution is that it did not undermine its social stability, and the main reason for the Ukrainian color revolution is that the driving force of Western countries is strong enough, and the specific political factors involved are unknown to us here.
But from the literature of previous people, we found such a passage:
\begin{center}
\indent \fbox{
    \parbox{.9\linewidth}{
    ~~~~The game between Russia and the United States. The outbreak of the ``Orange Revolution'' was largely due to the interference of external forces in Ukraine's internal affairs. There are four major participants: the United States, Russia, the European Union, and the ``New Europe'' countries - Poland and Lithuania, which are independent participants.\par

~~~~The fundamental difference between Western and Russian influence lies in different political mechanisms and social ideologies. The position of the United States and European countries on Ukraine is quite clear. 

~~$\bullet$~~Firstly, for more than a decade they have been conducting a targeted campaign to form a pro-Western Ukrainian elite, especially with the training of a large number of specialists and journalists, which turned out to be very effective. 

~~$\bullet$~~Secondly, officials in the United States and the European Union have taken extremely favorable positions to ensure transparency and democracy in the electoral process. 

~~$\bullet$~~Thirdly, in the situation of the Ukrainian crisis, the United States, the European Union, Poland and Lithuania smoothly became active mediators in the negotiations and provided sufficient information, diplomatic and special support for their actions. Western officials, while ensuring that their interests are realized, have also maintained their image of non-interference in the internal affairs of other countries. Thus, the pro-Western forces in Ukraine received substantial blessings.
    }
}
\end{center}

From the author's attitude, that is, the content, it is easy to see the main factors for the success of the Ukrainian color revolution, and we will not repeat it.
\newpage
\section{Conclusion}
\subsection{Strengths And Weaknesses}
Our research is basically based on the predecessors, and there are some improvements and shortcomings as follows:
\subsubsection{Strengths}
\begin{itemize}
    \item In the establishment of the index system of social stability, we find that the basic methods used by predecessors include analytic hierarchy process, questionnaire survey, Expert review, literature review and so on. We combine previous literature to determine the weight of each indicator will lead to our indicator system does not have a good standard. Here we use the unascertained measure theory model and introduce the concept of confidence degree, which can transform the weight of the index in different literatures into the more reliable weight of our index system, get the weight vector for the metric we need.
    \item Since there are 48 indicators in our indicator system, the correlation between indicators is difficult to give by correlation analysis. DEMATE is suitable for system analysis with many variables, complex relationships and unclear structure, and can also be used for the ordering of schemes. ISM can also be used. Here we boldly combine the two models, because we do not have high requirements for the relationship between variables, ISM can also effectively achieve this goal: the final Boolean matrix has only two values, true or false.
    \item Due to space limitations, we did not show the relationship matrix between indicators in the paper but turned it into the form of a heat map, which more intuitively shows the results of our model, and can directly find out whether there is a relationship between two factors in that figure.
\end{itemize}

\subsubsection{Weaknesses}
\begin{itemize}
    \item To a certain extent, the relationship between the elements of a system depends on people's experience. Insufficient objectivity Many of the papers we have consulted are not perfect in the selection of indicators, so the weight vectors obtained by our unknown measure theory may not be very accurate.
    \item There are too many indicators established by the indicator system, so we can only give the general relationship between indicators by using the DEMATEL-ISM method, but cannot draw a multi-level hierarchical directed topology diagram according to the relationship between indicators. Even if we give the relationship between them as a heat map, it is not intuitive to feel it in layers.
    \item Our analysis of the influencing factors of the color revolution is relatively superficial, because the political factors and diplomatic factors designed in it are full of unknowns, and the research of predecessors is seriously differentiated: it is reflected in the great disagreement on whether it is a color revolution.
    \item Because the concept of color revolution is not very clear, we can only select the most representative of them, and all parties have made social stability judgment and influencing factor analysis of the two cases identified as color revolutions.
    \item Due to the subjectivity of the social stability scores of the two color revolutions we selected, the subsequent analysis of the reasons for the change of regime caused by the color revolutions is also based on the discussion, and cannot give the actual influencing factors well. Of course, this is also beyond the scope on this paper.
 \end{itemize}
\subsection{Future Work}
With the world entering a new period of turbulence and change, the ``color revolution'' is still an important topic worth discussing. The western forces led by the United States have not abandoned the ``democratic export'', have not stopped the attempts to subvert the regimes of other countries or regions, and the threat will not disappear. The ``color revolution'' itself is also in the process of constant change. We should look at the problem from the perspective of development, observe and understand it with the times, and formulate corresponding preventive measures. 

Just as people could not predict the new changes brought by information technology to the ``color revolution'' more than a decade ago, it is not ruled out that the future ``color revolution'' will continue to change in connotation or take on unexpected new features.Our research on the color revolution needs to be further deepened.
\begin{itemize}
	\item The indicators that affect social stability are far more than that, we only list the representative, which inevitably duplicate or missing indicators, and China has undergone earth-shaking changes, and the future needs to update the indicator system and the weight of each indicator in real time to conform to the trend of the times.
	\item The social early warning model we established did not judge the trend of future development, but only found that the stability index changed periodically, and the future research in this area can continue to be deepened.
	\item In the future, it is also necessary to unify the division and judgment of color revolutions, so that the relevant color revolutions can be systematically analyzed more rigorously.
\end{itemize}























































% The instance of long and wide tables are shown in Table \ref{tb:longtable}.

% % 长表格示例,更多用法请参考 longtable 宏包文档
% % 以下环境及对应参数可实现表格内的自动换行与表格的自动断页
% % 您也可以选择自行载入 tabularx 宏包,并通过 X 参数指定对应列自动换行
% \begin{longtable}{ p{4em} p{14em} p{14em} }
% \caption{Basic Information about Three Main Continents (scratched from Wikipedia)}
% \label{tb:longtable}\\
% \toprule
% Continent & Description & Information \\
% \midrule
% Africa & Africa Continent is surrounded by the Mediterranean Sea to the
% north, the Isthmus of Suez and the Red Sea to the northeast, the Indian
% Ocean to the southeast and the Atlantic Ocean to the west. &
% At about 30.3 million km$^2$ including adjacent islands, it covers 6\%
% of Earth's total surface area and 20\% of its land area. With 1.3
% billion people as of 2018, it accounts for about 16\% of the world's
% human population. \\
% \midrule
% Asia & Asia is Earth's largest and most populous continent which
% located primarily in the Eastern and Northern Hemispheres.
% It shares the continental landmass of Eurasia with the continent
% of Europe and the continental landmass of Afro-Eurasia with both
% Europe and Africa. &
% Asia covers an area of 44,579,000 square kilometres, about 30\%
% of Earth's total land area and 8.7\% of the Earth's total surface
% area. Its 4.5 billion people (as of June 2019) constitute roughly
% 60\% of the world's population. \\
% \midrule
% Europe & Europe is a continent located entirely in the Northern
% Hemisphere and mostly in the Eastern Hemisphere. It comprises the
% westernmost part of Eurasia and is bordered by the Arctic Ocean to
% the north, the Atlantic Ocean to the west, the Mediterranean Sea to
% the south, and Asia to the east. &
% Europe covers about 10,180,000 km$^2$, or 2\% of the Earth's surface
% (6.8\% of land area), making it the second smallest
% continent. Europe had a total population of about 741 million (about
% 11\% of the world population) as of 2018. \\
% \bottomrule
% \end{longtable}
















% 以下为信件/备忘录部分,不需要可自行去掉
% 如有需要可将整个 letter 环境移动到文章开头或中间
% 请在第二个花括号内填写标题,如「信件」(Letter)或「备忘录」(Memorandum)
\begin{letter}{\huge{$\mathscr{MEMO}$}}
\begin{flushleft}  % 左对齐环境,无首行缩进
\textbf{To:} MCM/ICM organizing committee\\
\textbf{From:} Team MI00008\\
\textbf{Date:}\today\\
\textbf{Subject:} In order to prevent colorr evolution and maintain social stability
\end{flushleft}
\lettrine{T}{he} second ``color revolution'' in Ukraine shows that ``color revolution'' is probably not a concept in the historical past tense. Looking through the ``color revolution'', we can find that although its form has undergone two major transmutations in the past ten years, there are still clues to its connotation. No matter how the pattern changes, the ``color revolution'' is still an anti-government movement instigated by international powers against the target country or region according to their strategic interests, which take advantage of domestic political and economic contradictions and use the people's social resistance movement to try to overthrow the current regime of a country.

And not every country is helpless in the face of the encroachment of ``color revolutions''. Uzbekistan, Belarus and other countries have successfully resisted ``color revolutions''. From their experience, we can roughly summarize a few suggestions:

{\itshape \begin{enumerate}[0]
    \item[$\bullet$] Keep a firm grip on the local media to prevent public opinion from getting out of control. There is no substitute for the role of the media in regime change, and its entry into and occupation of a country's position of public opinion can help to bring down that country's regime. For the ruling party, therefore, to give ground. That means the beginning of the loss of power, public opinion out of control, not much time.

    \item[$\bullet$] Legislation restricting the activities of NGOs in the country. On November 23, 2005, the Russian Duma adopted a draft law on non-governmental organizations inside and outside Russia, "According to this bill, non-governmental organizations in Russia are not allowed to accept foreign funds to engage in political activities, and foreign non-governmental organizations cannot fund non-governmental organizations in Russia to engage in political activities, and will clean, rectify, and restrain the behavior of informal organizations at home and abroad by means of registration, review of charters, supervision of the source and flow of funds, and inspection of whether the activities of non-governmental organizations are consistent with their purposes.


    \item[$\bullet$]  It is necessary to handle the relationship with the opposition well. In the tactics of dealing with the opposition, countries that are not very tense in partisanship generally use the strategy of both unity and struggle. Kazakhstan, for example, on the one hand, has taken harsh measures against the excesses of some hardline opposition parties that disturb social order. On the other hand, ``in his address to Parliament in September 2005, the President stated that Kazakhstan planned to establish a 'State Commission for the Elaboration and Refinement of a Programme of Democratic Reform' under the direct authority of the President and to invite all political parties, public associations and non-governmental organizations to participate in order to expand the space for the opposition's political participation''.

	\item[$\bullet$] Strengthen control over powerful departments such as the military and police. The inaction of the military, especially the police force and the internal guard forces, when the ``color revolution'' took place, was actually a disguised ``initiative'', that is, supporting the opposition's attack on the current regime with actual actions.
\end{enumerate}}
% \item Keep a firm grip on the local media to prevent public opinion from getting out of control. There is no substitute for the role of the media in regime change, and its entry into and occupation of a country's position of public opinion can help to bring down that country's regime. For the ruling party, therefore, to give ground. That means the beginning of the loss of power, public opinion out of control, not much time.
% \item Legislation restricting the activities of NGOs in the country. On November 23, 2005, the Russian Duma adopted a draft law on non-governmental organizations inside and outside Russia, "According to this bill, non-governmental organizations in Russia are not allowed to accept foreign funds to engage in political activities, and foreign non-governmental organizations cannot fund non-governmental organizations in Russia to engage in political activities, and will clean, rectify, and restrain the behavior of informal organizations at home and abroad by means of registration, review of charters, supervision of the source and flow of funds, and inspection of whether the activities of non-governmental organizations are consistent with their purposes.
% \item It is necessary to handle the relationship with the opposition well. In the tactics of dealing with the opposition, countries that are not very tense in partisanship generally use the strategy of both unity and struggle. Kazakhstan, for example, on the one hand, has taken harsh measures against the excesses of some hardline opposition parties that disturb social order. On the other hand, ``in his address to Parliament in September 2005, the President stated that Kazakhstan planned to establish a 'State Commission for the Elaboration and Refinement of a Programme of Democratic Reform' under the direct authority of the President and to invite all political parties, public associations and non-governmental organizations to participate in order to expand the space for the opposition's political participation''.
% \item Strengthen control over powerful departments such as the military and police. The inaction of the military, especially the police force and the internal guard forces, when the ``color revolution'' took place, was actually a disguised ``initiative'', that is, supporting the opposition's attack on the current regime with actual actions.
% \end{itemize}
\end{letter}








% \begin{figure}[h!] 	
% \vspace{-0.8cm}   %调整图片与上文的垂直距离  
% \setlength{\abovecaptionskip}{0.cm} %调整标题上方的距离   
% \setlength{\abovecaptionskip}{0.cm} %调整标题下方的距离 	   
% \setlength{\belowdisplayskip}{0.pt} 	
% \includegraphics[width=15cm]{img/c1.pdf}  
% \end{figure}
% \newpage
% \begin{figure}[h!] 	
% \vspace{-0.8cm}   %调整图片与上文的垂直距离   
% \setlength{\belowdisplayskip}{10.pt} 	
% \includegraphics[width=15cm]{img/c2.pdf}  
% \end{figure}











\newpage
% 参考文献,此处以 MLA 引用格式为例
\begin{thebibliography}{100}

\bibitem{noauthor_chomage_nodate}
Chômage au sens du {BIT} et indicateurs sur le marché du travail (résultats
  de l’enquête emploi) ({BIT}) - troisième trimestre 2022.


\bibitem{cao_influence_2021}
Zhen Cao, Da-Peng Hao, Gang Tang, Zhi-Peng Xun, Hui Xia, and {School of
  Materials and Physics, China University of Mining and Technology, Xuzhou
  221116, China}.
\newblock Influence of cluster shaped defects on fracture process of fiber
  bundle.
\newblock {\it Acta Phys. Sin.}, 70(20):204602, 2021.

\bibitem{courchamp_100_2018}
Franck Courchamp and Corey J.~A. Bradshaw.
\newblock 100 articles every ecologist should read.
\newblock {\it Nature Ecology \& Evolution}, 2(2):395--401, February 2018.

\bibitem{farah_contacts_nodate}
Sabine Farah and Bastien Revel.
\newblock {CONTACTS} {LEAD} {MINISTRY}.

\bibitem{hutchinson_homage_1959}
G.~E. Hutchinson.
\newblock Homage to {Santa} {Rosalia} or {Why} {Are} {There} {So} {Many}
  {Kinds} of {Animals}?
\newblock {\it The American Naturalist}, 93(870,):145--159, 1959.


\bibitem{paine_food_1966}
Robert~T. Paine.
\newblock Food {Web} {Complexity} and {Species} {Diversity}.
\newblock {\it The American Naturalist}, 100(910):65--75, January 1966.


\bibitem{wenqi_research_2021}
Sun Wenqi, Luo Xuechun, Zhou Zaohong, and Shengfu~Beier Road.
\newblock {RESEARCH} {ON} {FACTORS} {AFFECTING} {SOCIAL} {STABILITY} {RISK}
  {BASED} {ON} {DEMEATEL}-{ISM}.
\newblock {\it Business and Management Research}, 5, 2021.

\bibitem{zhang_study_2022}
Zhishuo Zhang, Yao Xiao, Zitian Fu, Kaiyang Zhong, and Huayong Niu.
\newblock A {Study} on {Early} {Warnings} of {Financial} {Crisis} of {Chinese}
  {Listed} {Companies} {Based} on {DEA}–{SVM} {Model}.
\newblock {\it Mathematics}, 10(12):2142, June 2022.

\bibitem{__2013}
Dunwen Liu and Huaide Peng.
\newblock Safety risk evaluation of special long tunnel construction based on unconfirmed measure theory.
\newblock {\it Science \& Technology Review}, 31(15):31--34, 2013.
\newblock 12 citations(CNKI)[2023-2-5]{\textless}Peking University Core,
  CSCD{\textgreater}.

\bibitem{__2021}
Qian Liu.
\newblock {\it Study on the risk of social stability in Tibetan areas}.
\newblock Master, Southwest University of Political Science and Law, 2021.

\bibitem{__2021-2}
Lei Ming, Liu and An Qi, Chen.
\newblock
 Discuss the reasons and enlightenment of Belarus' successful response to the three "color revolutions".
\newblock {\it Chinese military conversion to civilian use}, (22):76--77, 2021.
\newblock 3 citations(CNKI)[2023-2-5].

\bibitem{__2008}
Peng Liu.
\newblock {\it
  Research on early warning mechanism and evaluation of public crisis in provincial central cities in China}.
\newblock Doctor, Harbin Engineering University, 2008.
\newblock 49 citations(CNKI)[2023-2-5].

\bibitem{_dematel-ism_2023}
Hongqi Lu, Shizhe Shao, Yanwei Liu, and Xiujuan Chen.
\newblock
  {Analysis of influencing factors of fire and explosion accidents in university laboratory based on DEMATEL} - {ISM}.
\newblock {\it 安全}, 44(1):23--31, 2023.

\bibitem{_dematel_2023}
Jun Song, Fang Li, and Zhong Hong Li
\newblock {Analysis of influencing factors of college students' emergency capacity based on DEMATEL method}.
\newblock {\it Technology and innovation management}, 44(1):90--96, 2023.

\bibitem{_px_2018}
Hai Jie, Yin, Rui Jiang, and Pei Yan, Lin.
\newblock {Analysis of the characteristics and influencing factors of social stability risk in PX project}.
\newblock {\it Engineering research - engineering in interdisciplinary perspective}, 10(03):276--287,
  2018.


\bibitem{__2017}
Jian Xu and Tian Xie
\newblock Social early warning analysis of international political turbulence
\newblock {\it Journal of Shanghai Jiaotong University (Philosophy and Social Sciences Edition)}, 25(3):23--33,
  2017.
\newblock {\textless}PKU Core, CSSCI{\textgreater}.



\bibitem{__2022-52}
\newblock {\it
 The change of Russian official cognition of "color revolution" and its response analysis}.
\newblock Master, East China Normal University, 2022.

\bibitem{__2020}
King Bin.
\newblock {\it Research on sharing finance and social stability}.
\newblock Master, People's Public Security University of China, 2020.



\bibitem{__2014}
Xiaoliang Zhu.
\newblock {\it Study on the risk assessment of social stability in china}.
\newblock Doctor, Central China Normal University, 2014.
\newblock 12 citations(CNKI)[2023-2-5].


\bibitem{__2019-1}
Yu Meitong
\newblock {\it Research on social stability index and its index system under risk society}.
\newblock Master, Zhejiang University, 2019.
\newblock 5 citations(CNKI)[2023-2-3].


\bibitem{__2019}
Chang Su.
\newblock
 Analysis of the factors affecting the social stability of Central Asian countries -- based on the comprehensive analysis of basic and impact factors.
\newblock {\it Research on Russia, Eastern Europe and Central Asia}, (3):61--83+156--157, 2019.
\newblock 8 citations(CNKI)[2023-2-3]{\textless}CSSCI, AMI{\textgreater}.



\bibitem{__2003}
Wei Zhideng.
\newblock Reflections on the early warning mechanism of social risks.
\newblock {\it Social Science}, (7):65--71, 2003.
\newblock 159 citations(CNKI)[2023-2-5]{\textless}PKU Core,
  CSSCI{\textgreater}.


\bibitem{__2007}
Yongzhong Wei and Shaozhong Wu
\newblock Discussion on the establishment of early warning level model for social security and stability in China.
\newblock {\it Public Security Studies}, (1):32--38, 2007.
\newblock 20 citations(CNKI)[2023-2-3].

\end{thebibliography}



% 以下为附录内容
% 如您的论文中不需要附录,请自行删除
% \begin{subappendices}  % 附录环境
% \section{Appendix}


% \end{subappendices}  % 附录内容结束



















































\end{document}  % 结束
